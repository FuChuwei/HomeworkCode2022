\question
\qpart
假设位于$x$的消费者在厂商1 和厂商2的消费效用相同,则有:
\begin{align}
     u_{1,x}&=u_{2,x} \notag \\ 
     v_1-tx-p_1 &= v2-t(1-x)-p_2 \notag \\
     \therefore \  x &= \frac{v_1-v_2+p_2-p_1+t}{2t}
\end{align}
则位于[o,x)上的消费者在厂商1的消费效用大于在厂商2的消费效用,因此都会选择去厂商1; 位于(x,1]上的消费者在厂商2的消费效用大于在厂商2的消费效用,因此都会选择去厂商2。则厂商1和厂商2的利润$\pi_1$和$\pi_2$满足:
\begin{align}
    \pi_1 &= p_1x = p_1\frac{v_1-v_2+p_2-p_1+t}{2t}  \notag \\
    \pi_2 &= p_2(1-x) = p_2\frac{v_2-v_1+p_1-p_2+t}{2t}   \notag
\end{align}
使厂商1和厂商2利润最大化的价格$p_1^{\star}$和$p_2^{\star}$应满足一阶条件等于0,即:
\begin{align}
    \frac{\partial \pi_1 }{\partial p_1} &=\frac{v_1-v_2+p_2-2p_1^{\star}+t}{2t} = 0 \notag \\
    \therefore \ p_2-2p_1^{\star} &= v_2-v_1-t \\
    \frac{\partial \pi_2 }{\partial p_2} &=\frac{v_2-v_1+p_1-2p_2^{\star}+t}{2t} = 0 \notag \\
    \therefore \ p_1-2p_2^{\star} &= v_1-v_2-t 
\end{align}
联立式(1.2)和式(1.3),可得均衡的$(p_1^{\star},p_2^{\star})$:
\begin{equation}
  \begin{cases}
p_1^{\star} = t+\frac{v_1-v_2}{3} \\
p_2^{\star} = t+\frac{v_2-v_1}{3}
\end{cases}  
\end{equation}

\qpart
不妨设厂商1和厂商2的定价分别为$p_1(x)$和$p_2(x)$($p_1(x)\geq0$,$p_2(x)\geq 0$),是消费者所在位置$x$的函数,并且满足:
\begin{equation}
   \begin{cases}
    \frac{\partial p_1(x)}{\partial x}&\geq0 \\
    \frac{\partial p_2(x)}{\partial x}&\leq0
\end{cases} 
\end{equation}


即对于厂商1而言,x越大意味着消费者距离越远,交通成本越高,为了吸引消费者x会降价,因此$p_1(x)$随着$x$的增加而增加;
对于厂商2而言,x越大意味着消费者距离越近,交通成本越低,因此$p_2(x)$随着$x$的增加而减少。

同样的,假设位于$x$的消费者在厂商1 和厂商2的消费效用无差异,则有:
\begin{align}
     u_{1,x}&=u_{2,x} \notag \\ 
     v_1-tx-p_1(x) &= v2-t(1-x)-p_2(x) \notag \\
     \therefore \  x &= \frac{v_1-v_2+p_2(x)-p_1(x)+t}{2t}
\end{align}
则厂商1和厂商2的利润$\pi_1$和$\pi_2$满足:
\begin{align}
    \pi_1 &=\int_0^x p_1(x)dx \\
    \pi_2 &= \int_x^1 p_2(x)dx 
\end{align}
不妨设$F_1(x)$和$F_2(x)$满足:
\begin{equation*}
    F_i^{'} (x) = p_i(x)
\end{equation*}
则式(1.7)和式(1.8)可以写为:
\begin{align}
    \pi_1(x) &=F_1(x)-F_1(0) \\
    \pi_2(x) &=F_2(1)-F_2(x)
\end{align}
利润最大化下,应满足:
\begin{align}
    \frac{\partial \pi_i(x) }{\partial p_i} &=\frac{\partial \pi_i(x)}{\partial x}\bigg/\frac{\partial p_i(x)}{\partial x}=0 
\end{align}
分母$\frac{\partial p_i(x)}{\partial x}$不为0,因此有:
\begin{align}
    \frac{\partial \pi_i(x)}{\partial x} &=0 \notag\\
    \therefore &\begin{cases}
      \frac{\partial \pi_1(x)}{\partial x} = p_1(x) =0 \\
      \frac{\partial \pi_2(x)}{\partial x} = -p_2(x) =0
\end{cases}  \notag \\
    \therefore x &= \frac{v_1-v_2+t}{2t} 
\end{align}
又因为$p_2$在$[0,x)$上单调递增且,$p_2(x)=0$,所以在$[0,x)$上,$p_{2,x}\equiv0$;同理,在$(x,1]$上,$p_{1,x}\equiv0$。

因此,在$[0,x)$上有:
\begin{align}
    u_1 \geq u_2 & \ and \ p_{2,x}=0 \notag \\
    v_1-tx-p_1(x) &\geq v_2-t(1-x)-p_2(x) \notag \\
    v_1-tx-p_1(x) &\geq v_2-t(1-x) \notag \\
   0\leq p_1(x) & \leq v_1-v_2+t-2tx 
\end{align}
为使利润尽可能地高,厂商应尽量提高价格因此有:
\begin{align}
    u_1,0=v_1&-p_1^{\star}(0)=0 \notag \\
    \therefore p_1^{\star}(0) = v_1 \\
    u_2,1=v_2&-p_2^{\star}(1)=0 \notag \\
    \therefore p_2^{\star}(1) = v_2 
\end{align}
将式(1.14)、式(1.15)和$p_i(x)=0$代入式(1.5)可得均衡定价$(p_1^{\star},p_2^{\star})$:
\begin{align}
\begin{cases}
    p_1^{\star}(x) &= \frac{2tv_1}{v_2-v_1-t}x+v_1 \\
    p_2^{\star}(x) &= \frac{2tv_2}{v_2-v_1+t}x+\frac{v_2(v_2-v_1-t)}{v_2-v_1+t}
\end{cases}
\end{align}

\qpart












\begin{align}
    \pi_1 &=\int_0^{\frac{p_2-p_1+t}{2t}} p_1f(x)dx   \notag \\
    &=\int_0^{\frac{p_2-p_1+t}{2t}} p_1 x^{a-1}dx \notag \\
    &=\left.\frac{p_1}{a}x^a \right|_0^{\frac{p_2-p_1+t}{2t}} \notag \\
    &=\frac{p_1}{a}(\frac{p_2-p_1+t}{2t})^a\\
    \pi_2 &=\int_{\frac{p_2-p_1+t}{2t}}^1 p_2f(x)dx  \notag \\
    &=\int_{\frac{p_2-p_1+t}{2t}}^1 p_2 x^{a-1}dx \notag \\
    &=\left.\frac{p_2}{a}x^a \right|_{\frac{p_2-p_1+t}{2t}}^1 \notag \\
    &=\frac{p_2}{a} -\frac{p_2}{a}(\frac{p_2-p_1+t}{2t})^a 
\end{align}

使厂商1和厂商2利润最大化的价格$p_1^{\star}$和$p_2^{\star}$应满足一阶条件等于0,即:
\begin{align}
    \frac{\partial \pi_1 }{\partial p_1} &=\frac{1}{a}(\frac{p_2-p_1^{\star}+t}{2t})^a -\frac{p_1^{\star}}{2t}(\frac{p_2-p_1^{\star}+t}{2t})^{a-1} =0\notag \\
    &=(p_2-p_1^{\star}+t)-ap_1^{\star} =0 \notag \\
    \therefore \ p_2- &(a+1)p_1^{\star}= -t \\
    \frac{\partial \pi_2 }{\partial p_2} &=\frac{1}{a}-\frac{1}{a}(\frac{p_2^{\star}-p_1+t}{2t})^a -\frac{p_2^{\star}}{2t}(\frac{p_2^{\star}-p_1+t}{2t})^{a-1} =0 \notag \\
    &=1-(\frac{p_2^{\star}-p_1+t}{2t})^a-\frac{ap_2^{\star}}{p_2^{\star}-p_1+t}(\frac{p_2^{\star}-p_1+t}{2t})^a=0 \notag \\
    &=\left[(\frac{2t}{p_2^{\star}-p_1+t})^a-\frac{(a+1)p_2^{\star}-p_1+t}{p_2^{\star}-p_1+t}\right](\frac{p_2^{\star}-p_1+t}{2t})^a=0 
\end{align}
若式(1.29)第二项等于0,则:
\begin{align}
    p_2^{\star}-p_1+t &= 0 \notag \\
    p_2^{\star}-p_1&=v_2-v_1-t 
\end{align}
将式(1.30)与式(1.28)联立,得:
\begin{align}
    \begin{cases}
     p_1^{\star} = 0 \\
     p_2^{\star} = v_2-v_1-t  \\
    \end{cases}
\end{align}
因为$v_2<v_1$,所以$p_2^{\star}<0$,这与$p_2\geq0$的假设矛盾,不可取。

若式(1.29)第一项等于0,则:
\begin{align}
    (\frac{2t}{p_2^{\star}-p_1+t})^a-\frac{(a+1)p_2^{\star}-p_1+t}{p_2^{\star}-p_1+t} = 0
\end{align}
将式(1.32)与式(1.28)联立,得: