\qsubpart 
假设位于$x$的消费者在厂商1 和厂商2的消费效用相同,则有:
\begin{align}
     u_{1,x}&=u_{2,x} \notag \\ 
     v_1-tx-p_1 &= v2-t(1-x)-p_2 \notag \\
     \therefore \  x &= \frac{p_2-p_1+t}{2t}
\end{align}

则位于$[0,\frac{p_2-p_1+t}{2t})$上的消费者都会选择去厂商1; 位于$(\frac{p_2-p_1+t}{2t},1]$上的消费者都会选择去厂商2。则厂商1和厂商2的利润$\pi_1$和$\pi_2$满足:
\begin{align}
    \pi_1 &=\int_0^{\frac{p_2-p_1+t}{2t}} p_1f(x)dx   \notag \\
    &=\int_0^{\frac{p_2-p_1+t}{2t}} p_1 x^{a-1}dx \notag \\
    &=\left.\frac{p_1}{a}x^{\alpha} \right|_0^{\frac{p_2-p_1+t}{2t}} \notag \\
    &=\frac{p_1}{a}(\frac{p_2-p_1+t}{2t})^{\alpha}\\
    \pi_2 &=\int_{\frac{p_2-p_1+t}{2t}}^1 p_2f(x)dx  \notag \\
    &=\int_{\frac{p_2-p_1+t}{2t}}^1 p_2 x^{a-1}dx \notag \\
    &=\left.\frac{p_2}{a}x^{\alpha} \right|_{\frac{p_2-p_1+t}{2t}}^1 \notag \\
    &=\frac{p_2}{a} -\frac{p_2}{a}(\frac{p_2-p_1+t}{2t})^{\alpha} 
\end{align}

使厂商1和厂商2利润最大化的价格$p_1^{\star}$和$p_2^{\star}$应满足一阶条件等于0,即:
\begin{align}
    \frac{\partial \pi_1 }{\partial p_1} &=\frac{1}{a}(\frac{p_2-p_1^{\star}+t}{2t})^{\alpha} -\frac{p_1^{\star}}{2t}(\frac{p_2-p_1^{\star}+t}{2t})^{a-1} =0\notag \\
    &=(p_2-p_1^{\star}+t)-ap_1^{\star} =0 \notag \\
    \therefore \ p_2- &(a+1)p_1^{\star}= v_2-v_1-t \\
    \frac{\partial \pi_2 }{\partial p_2} &=\frac{1}{a}-\frac{1}{a}(\frac{p_2^{\star}-p_1+t}{2t})^{\alpha} -\frac{p_2^{\star}}{2t}(\frac{p_2^{\star}-p_1+t}{2t})^{a-1} =0 \notag \\
    &=1-(\frac{p_2^{\star}-p_1+t}{2t})^{\alpha}-\frac{ap_2^{\star}}{p_2^{\star}-p_1+t}(\frac{p_2^{\star}-p_1+t}{2t})^{\alpha}=0 \notag \\
    &=\left[(\frac{2t}{p_2^{\star}-p_1+t})^{\alpha}-\frac{(a+1)p_2^{\star}-p_1+t}{p_2^{\star}-p_1+t}\right](\frac{p_2^{\star}-p_1+t}{2t})^{\alpha}=0 
\end{align}
若式(1.29)第二项等于0,则:
\begin{align}
    p_2^{\star}-p_1+t &= 0 \notag \\
    p_2^{\star}-p_1&=v_2-v_1-t 
\end{align}
将式(1.30)与式(1.28)联立,得:
\begin{align}
    \begin{cases}
     p_1^{\star} = 0 \\
     p_2^{\star} = v_2-v_1-t  \\
    \end{cases}
\end{align}
因为$v_2<v_1$,所以$p_2^{\star}<0$,这与$p_2\geq0$的假设矛盾,不可取。

若式(1.29)第一项等于0,则:
\begin{align}
    (\frac{2t}{p_2^{\star}-p_1+t})^{\alpha}-\frac{(a+1)p_2^{\star}-p_1+t}{p_2^{\star}-p_1+t} = 0
\end{align}
将式(1.32)与式(1.28)联立,得:



同样的,假设第一问的消费者剩余为$S^{\alpha}$,均衡点(消费者效用无差异)为$x_0^{\alpha}$,均衡价格为$(p_1^{\star(\alpha)},p_2^{\star(\alpha)})$;第二问的消费者剩余为$S^{\beta}$,均衡点)为$x_0^{\beta}$,均衡价格为$(p_1^{\star(\beta)},p_2^{\star(\beta)})$:
\begin{align}
    S^{\alpha} &= \int_0^{x_0^{\alpha}}(p_1^m-p_1^{\star(\alpha)})dx+\int_{x_0^{\alpha}}^{1}(p_2^m-p_2^{\star(\alpha)})dx \\
    S^{\beta} &= \int_0^{x_0^{\beta}}(p_1^m-p_1^{\star(\beta)})dx+\int_{x_0^{\beta}}^{1}(p_2^m-p_2^{\star(\beta)})dx 
\end{align}
将式(1.1)、式(1.4)、式(1.12)及(1.16-1.19)代入式(1.20)与式(1.21)整理得到: