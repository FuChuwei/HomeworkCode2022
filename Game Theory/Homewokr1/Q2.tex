
\question
\textbf{厂商的Nash均衡定价为$v$。}

对所有厂商而言,在第$k$次被消费者搜寻到并询价的概率$f(k)$均为:
\begin{align}
     f(k) &= \frac{\binom{n-1}{k-1}(k-1)!}{\binom{n}{k}(k)!}  \notag\\   
     &=\frac{1}{n}
\end{align}
假设厂商i的定价为$p_i$,在第$k$次搜索询价中商品被购买,那么其收益为$w_{i,k}$,期望收益为$E(w_{i,k})$,有:
\begin{align}
    w_{i,k} &= p_i \notag \\ 
    E(w_{i,k}) &= f(k)w_{i,k} +0 \notag \\
    &=\frac{p_i}{n}
\end{align}


对于消费者而言,在询价后有三种状态:购买、退出、继续搜索并询价。其中,继续搜索并询价只能是中间状态,即总是存在$l\in [1,n-1]$,在$l$次询价并搜索后,消费者会在第$l+1$次询价结束后要么选择购买,要么选择退出市场,因此接下来只讨论购买和退出时消费者的策略。

假设消费者的效用为$u$,期望效用为$E(u)$。


\textbf{假设消费者在第$k$次搜索询价中购买了厂商i的商品:}

消费者的效用为:
\begin{equation}
    u_{k,i}=v-(k-1)s-p_i
\end{equation}
那么,消费者对在第$k$次询价搜索中购买商品的期望效用即为:
\begin{align}
    E(u_k) &= \sum_{i=1}^{n}u_{k,i}f(k) \notag\\
    &=\frac{1}{n}\left[n(v+s-ks)-\sum_{i=1}^np_i\right] 
\end{align}
由于消费者不参与定价,则要使自己的期望效用最大,只能选择最优的搜寻次数$k^{\star}$:
\begin{align}
    k^{\star} &= \mathop{\arg\max}\limits_k E(u_k) \notag\\
    &= \mathop{\arg\max}\limits_k \frac{1}{n}\left[n(v+s-ks)-\sum_{i=1}^np_i\right] \notag \\
    &=\mathop{\arg\max}\limits_k \ v+s-ks
\end{align}
式(2.5)第二行到第三行的变换是因为$\sum_{i=1}^np_i$与$k$的取值无关,所以可以忽略。


显然,式(2.5)关于$k$单调递减,因此无论厂商如何定价,对消费者而言要想获得最优期望效用最好的搜索次数是$k^{\star}=1$,即在进入市场的第一次搜索中就购买。


\textbf{接下来,考虑消费者退出市场的情况:假设消费者在第$k$次搜寻询价后就退出了市场:}

此时,消费者的期望效用为:
\begin{align}
    E(u_k) = u_k = -(k-1)s
\end{align}
显然,式(2.6)也是关于$k$单调递减,因此要使自己的期望效用最大,消费者应当在第一次搜索询价后立即退出,即$k^{\star}=1$。

综上,无论消费者最终是要购买商品还是退出市场,消费者最优的搜索询价次数都是$k^{\star}=1$。

\textbf{最后,考虑在消费者最好的策略给定后,厂商的策略:}

因为无论厂商如何定价,消费者总是在第一次搜索询价时购买或退出能获得最好的期望效用,因此,若厂商想要将商品售出,则厂商i的定价$p_i$应当满足:
\begin{align}
    v - p_i \geq 0
\end{align}
因为厂商i的期望收益关于$p_i$单调递增,厂商有充足的动力将价格定在消费者能接受的最高价格上,也就是$p_i^{\star}=p^{\star}=v$。

需要注意的是,当$p_i^{\star}=p^{\star}=v$时,消费者的效用为零,此时消费者既有可能退出市场,也有可能购买商品。因此,有必要对上述均衡是否存在再做进一步的说明:\textbf{即是否存在某一个厂商j为了提高消费者购买商品的几率而改变价格(如将价格定为$p_j = p^{\star}-\epsilon$,$\epsilon$是一个极小正值)。}

从生活经验上说,当消费者的效用越大,选择购买的概率越高,否则就没有讨论的必要。因此不妨假定消费者在第一次搜价时若$u>0$则一定会购买商品,那么厂商j的期望收益为:
\begin{align}
    E(w_j)&=\frac{1}{n}p_j \notag \\
    &=\frac{p^{\star}}{n}-\frac{\epsilon}{n}
\end{align}

而对于其他厂商,由于将价格定在$p^{\star}=v$上,其期望收益为:
\begin{align}
    E(w_{-j})&=\frac{1}{n}p^{\star}g(0) \notag \\
    &=\frac{p^{\star}}{n}g(0)
\end{align}
其中g(0)是当效用为0时消费者选择购买商品的概率,取值范围为[0,1]。

若$p^{\star}=v$是Nash均衡,即当其他厂商的定价为$v$时,任意厂商的最优定价也是$v$,而没有动力去改变,则应当满足:
\begin{align}
    E(w_{-j})&\geq  E(w_j) \notag \\
    \frac{p^{\star}}{n}g(0) &\geq \frac{p^{\star}}{n}-\frac{\epsilon}{n} \notag \\
    g(0)&\geq 1-\frac{\epsilon}{p^{\star}} 
\end{align}
又因为$\epsilon$是任意可取的正值,因此若$p^{\star}$是Nash均衡,则$g(0)\geq1$,又因为$g(0)\leq 1$,必有:
\begin{align}
    g(0) = 1
\end{align}
即,当且仅当消费者在效用为零时也一定会购买商品时,才会存在Nash均衡定价,此时Nash均衡定价为$p^{\star}=v$。












