\question
\qpart
假设位于$x$的消费者在厂商1 和厂商2的消费效用相同,则有:
\begin{align}
     u_{1,x}&=u_{2,x} \notag \\ 
     v_1-tx-p_1 &= v_2-t(1-x)-p_2 \notag \\
     \therefore \  x &= \frac{v_1-v_2+p_2-p_1+t}{2t}
\end{align}
则位于[o,x)上的消费者在厂商1的消费效用大于在厂商2的消费效用,因此都会选择去厂商1; 位于(x,1]上的消费者在厂商2的消费效用大于在厂商2的消费效用,因此都会选择去厂商2。则厂商1和厂商2的利润$\pi_1$和$\pi_2$满足:
\begin{align}
    \pi_1 &= p_1x = p_1\frac{v_1-v_2+p_2-p_1+t}{2t}  \notag \\
    \pi_2 &= p_2(1-x) = p_2\frac{v_2-v_1+p_1-p_2+t}{2t}   \notag
\end{align}
使厂商1和厂商2利润最大化的价格$p_1^{\star}$和$p_2^{\star}$应满足一阶条件等于0,即:
\begin{align}
    \frac{\partial \pi_1 }{\partial p_1} &=\frac{v_1-v_2+p_2-2p_1^{\star}+t}{2t} = 0 \notag \\
    \therefore \ p_2-2p_1^{\star} &= v_2-v_1-t \\
    \frac{\partial \pi_2 }{\partial p_2} &=\frac{v_2-v_1+p_1-2p_2^{\star}+t}{2t} = 0 \notag \\
    \therefore \ p_1-2p_2^{\star} &= v_1-v_2-t 
\end{align}
联立式(1.2)和式(1.3),可得均衡的$(p_1^{\star},p_2^{\star})$:
\begin{equation}
  \begin{cases}
p_1^{\star} = t+\frac{v_1-v_2}{3} \\
p_2^{\star} = t+\frac{v_2-v_1}{3}
\end{cases}  
\end{equation}

\qpart
不妨设厂商1和厂商2的定价分别为$p_1(x)$和$p_2(x)$($p_1(x)\geq0$,$p_2(x)\geq 0$),是消费者所在位置$x$的函数,并且满足:
\begin{equation}
   \begin{cases}
    \frac{\partial p_1(x)}{\partial x}&\leq0 \\
    \frac{\partial p_2(x)}{\partial x}&\geq0
\end{cases} 
\end{equation}


即对于厂商1而言,x越大意味着消费者距离越远,交通成本越高,为了吸引消费者x会降价,因此$p_1(x)$随着$x$的增加而减少;
对于厂商2而言,x越大意味着消费者距离越近,交通成本越低,因此$p_2(x)$随着$x$的增加而增加。

同样的,假设位于$x$的消费者在厂商1 和厂商2的消费效用无差异,则有:
\begin{align}
     u_{1,x}&=u_{2,x} \notag \\ 
     v_1-tx-p_1(x) &= v_2-t(1-x)-p_2(x) \notag \\
     \therefore \  x &= \frac{v_1-v_2+p_2(x)-p_1(x)+t}{2t}
\end{align}
则厂商1和厂商2的利润$\pi_1$和$\pi_2$满足:
\begin{align}
    \pi_1 &=\int_0^x p_1(x)dx \\
    \pi_2 &= \int_x^1 p_2(x)dx 
\end{align}
不妨设$F_1(x)$和$F_2(x)$满足:
\begin{equation*}
    F_i^{'} (x) = p_i(x)
\end{equation*}
则式(1.7)和式(1.8)可以写为:
\begin{align}
    \pi_1(x) &=F_1(x)-F_1(0) \\
    \pi_2(x) &=F_2(1)-F_2(x)
\end{align}
利润最大化下,应满足:
\begin{align}
    \frac{\partial \pi_i(x) }{\partial p_i} &=\frac{\partial \pi_i(x)}{\partial x}\bigg/\frac{\partial p_i(x)}{\partial x}=0 
\end{align}
分母$\frac{\partial p_i(x)}{\partial x}$不为0,因此有:
\begin{align}
    \frac{\partial \pi_i(x)}{\partial x} &=0 \notag\\
    \therefore &\begin{cases}
      \frac{\partial \pi_1(x)}{\partial x} = p_1(x) =0 \\
      \frac{\partial \pi_2(x)}{\partial x} = -p_2(x) =0
\end{cases}  \notag \\
    \therefore x &= \frac{v_1-v_2+t}{2t} 
\end{align}
又因为$p_2$在$[0,\frac{v_1-v_2+t}{2t})$上单调递增且,$p_2(x)=0$,所以在$[0,\frac{v_1-v_2+t}{2t})$上,$p_{2,x}\equiv0$;同理,在$(\frac{v_1-v_2+t}{2t},1]$上,$p_{1,x}\equiv0$。

因此,在$[0,\frac{v_1-v_2+t}{2t})$上有:
\begin{align}
    u_1 \geq u_2 & \ and \ p_{2,x}=0 \notag \\
    v_1-tx-p_1(x) &\geq v_2-t(1-x)-p_2(x) \notag \\
    v_1-tx-p_1(x) &\geq v_2-t(1-x) \notag \\
   0\leq p_1(x) & \leq v_1-v_2+t-2tx 
\end{align}
为使利润尽可能地高,厂商应尽量提高价格因此有:
\begin{equation}
    p_1(x) = v_1-v_2+ t-2tx 
\end{equation}

同理,在$(\frac{v_1-v_2+t}{2t},1]$上,厂商2的价格满足:
\begin{equation}
    p_2(x) =  v_2-v_1-t+2tx 
\end{equation}

综上,在均衡时,价格$(p_1^{\star},p_2^{\star})$为:
\begin{equation}
p_1^{\star}(x)=
    \begin{cases}
      &  v_1-v_2+t-2tx,  \ x \in [0,\frac{v_1-v_2+t}{2t}] \\
     &0 ,\  x \in ( \frac{v_1-v_2+t}{2t},1]
    \end{cases}
\end{equation}

\begin{equation}
p_2^{\star}(x)= 
    \begin{cases}
     & 0,  \ x \in [0,\frac{v_1-v_2+t}{2t}] \\
     & v_2-v_1-t+2tx ,\  x \in ( \frac{v_1-v_2+t}{2t},1]
    \end{cases}
\end{equation}

\qpart
对于消费者$x$,所能接受的最大价格满足$u_1=0, u_2=0  $,因此最大价格$p^m_{1,x}$,$p^m_{2,x}$等价于:
\begin{align}
&p_1^m=v_1-t x \\
&p_2^m=v_2-t(1-x)
\end{align}

第一问的消费者剩余为$S^a$,均衡点(消费者效用无差异)为$x_0^a$,均衡价格为$(p_1^{\star(a)},p_2^{\star(a)})$;第二问的消费者剩余为$S^b$,均衡点)为$x_0^b$,均衡价格为$(p_1^{\star(b)},p_2^{\star(b)})$:
\begin{align}
    S^a &= \int_0^{x_0^a}(p_1^m-p_1^{\star(a)})dx+\int_{x_0^a}^{1}(p_2^m-p_2^{\star(a)})dx \\
    S^b &= \int_0^{x_0^b}(p_1^m-p_1^{\star(b)})dx+\int_{x_0^b}^{1}(p_2^m-p_2^{\star(b)})dx 
\end{align}
将式(1.1)、式(1.4)、式(1.12)及(1.16-1.19)代入式(1.20)与式(1.21)整理得到:
\begin{align}
S^a&=\int_0^{x_0^a}\left(v_1-t x-t-\frac{v_1-v_2}{3}\right) d x+\int_{x_0^a}^1\left(v_2-t(1-x)-t-\frac{v_2-v_1}{3}\right) d x \notag \\
&=\int_0^{x_0^a}\left[-t x+\frac{2 v_1+v_2-3 t}{3}\right] d x+\int_{x_0^a}^1\left[t x+\frac{2 v_2+v_1-6 t}{3}\right] d x \notag \\
&=-\frac{t}{2} x^2+\left.\frac{2 v_1+v_2-3 t}{3} x\right|_0 ^{x_0^a}+\frac{t x^2}{2}+\left.\frac{2 v_2+v_1-6 t}{3} x\right|_{x_0^a}^1 \notag \\
&=-\frac{t}{2}\left(x_0^a\right)^2+\frac{2 v_1+v_2-3 t}{3} x_0^a+\frac{t}{2}+\frac{2 v_2+v_1-6 t}{3}-\frac{t}{2}\left(x_0^a\right)^2-\frac{2 v_2+v_1-6t}{3} x_0^a \notag \\
&=-t\left(x_0^a\right)^2+\frac{v_1-v_2+3 t}{3} x_0^a+\frac{t}{2}+\frac{2 v_2+v_1-6 t}{3} \notag \\
&=\frac{\left(v_1-v_2+3 t\right)^2}{36 t}+\frac{2 v_2+v_1}{3}-\frac{3t}{2}
\end{align}

\begin{align}
 S^b&=\int_0^{x_0^b}\left(v_1-t x-v_1+v_2-t+2 t x\right) d x+\int_{x_0^b}^1\left(v_2-t(1-x)-v_2+v_1-2 t x\right) d x \notag \\
&=\int_0^{x_0^b}\left(v_2-t+t x\right) d x+\int_{x_0^b}^1\left(v_1-t x\right) d x \notag\\
&= \left.\frac{t}{2} x^2+\left(v_2-t\right)  \right|_0^{x_0^b}+\left.\left(-\frac{t}{2} x^2\right)+v_1 x\right|_{x_0^b} ^1 \notag\\
&= \frac{t}{2}\left(x_0^b\right)^2+\left(v_2-t\right) x_0^b-\frac{t}{2}+v_1+\frac{t}{2}\left(x_0^b\right)^2-v_1 x_0^b \notag\\
&=t\left(x_0^b\right)^2+\left(v_2-v_1-t\right) x_0^b+v_1-\frac{t}{2} \notag\\
&=\frac{\left(v_1-v_2+t\right)^2}{4 t}+\frac{\left(v_2-v_1-t\right)\left(v_1-v_2+t\right)}{2 t}+v_1-\frac{t}{2} \notag \\
&=-\frac{\left(v_1-v_2+t\right)^2}{4 t}+v_1-\frac{t}{2} .
\end{align}
当消费者总剩余减少时,有$S^b\textless S^a$:
\begin{align}
&\frac{\left(v_1-v_2+3 t\right)^2}{36 t}+\frac{2 v_2+v_1}{3}-\frac{3 t}{2}+\frac{\left(v_1-v_2+t\right)^2}{4 t}-v_1+\frac{t}{2}>0 \notag\\
&\frac{\left(v_1-v_2\right)+6 t\left(v_1-v_2\right)+9 t^2+9\left(v_1-v_2\right)^2+18\left(v_1-v_2\right) t+9 t^2}{36 t}+\frac{2\left(v_2-v_1\right)}{3}-t>0 . \notag \\
&\frac{10\left(v_1-v_2\right)^2+24 t\left(v_1-v_2\right)+18 t^2}{36 t}+\frac{2\left(v_2-v_1\right)}{3}-t>0 \notag \\
&\frac{5\left(v_1-v_2\right)^2+12 t\left(v_1-v_2\right)+9 t^2-12\left(v_1-v_2\right) t-18 t^2}{18 t}>0 \notag \\
& \qquad \ \qquad \ \qquad  \qquad  \qquad \ \qquad \ 5\left(v_1-v_2\right)^2>9 t^2
\end{align}
因此,当$v_1$、$v_2$和$t$满足式(1.24)时,价格歧视降低了消费者总剩余。
\qpart
\qsubpart 
在$v_1=v_2$条件下,
假设位于$x$的消费者在厂商1 和厂商2的消费效用相同,则有:
\begin{align}
     u_{1,x}&=u_{2,x} \notag \\ 
     v_1-tx-p_1 &= v_2-t(1-x)-p_2 \notag \\
     \therefore \  x &= \frac{p_2-p_1+t}{2t}
\end{align}

则位于$[0,\frac{p_2-p_1+t}{2t})$上的消费者都会选择去厂商1; 位于$(\frac{p_2-p_1+t}{2t},1]$上的消费者都会选择去厂商2。则厂商1和厂商2的利润$\pi_1$和$\pi_2$满足:

\begin{align}
    \pi_1 &=p_1F(\frac{p_2-p_1+t}{2t})   \notag \\
    &=p_1(\frac{p_2-p_1+t}{2t})^a \\
    \pi_2 &=p_2\left[1-F(\frac{p_2-p_1+t}{2t}) \right]  \notag \\
    &=p_2\left[1-(\frac{p_2-p_1+t}{2t})^a\right] 
\end{align}

使厂商1和厂商2利润最大化的价格$p_1^{\star}$和$p_2^{\star}$应满足一阶条件等于0,即:
\begin{align}
    \frac{\partial \pi_1 }{\partial p_1} &=(\frac{p_2-p_1^{\star}+t}{2t})^a -\frac{ap_1^{\star}}{2t}(\frac{p_2-p_1^{\star}+t}{2t})^{a-1} =0\notag \\
    &p_1^{\star}=\frac{p_2+t}{a+1} \\
    \frac{\partial \pi_2 }{\partial p_2} 
    &=1-(\frac{p_2^{\star}-p_1+t}{2t})^a-\frac{ap_2^{\star}}{p_2^{\star}-p_1+t}(\frac{p_2^{\star}-p_1+t}{2t})^a=0 \notag \\
    &=\left[(\frac{2t}{p_2^{\star}-p_1+t})^a-\frac{(a+1)p_2^{\star}-p_1+t}{p_2^{\star}-p_1+t}\right](\frac{p_2^{\star}-p_1+t}{2t})^a=0 
\end{align}

当$a=0$,则式(1.29)$\equiv 0 $,则无论$p_2$为何值,都可以使厂商2的利润最大,因此没有稳定的均衡,不可取。

当$a>0$,若式(1.29)第二项等于0,则:
\begin{align}
    p_2^{\star}-p_1+t &= 0 
\end{align}
将式(1.30)与式(1.28)联立,得:
\begin{align}
    \begin{cases}
     p_1^{\star} = 0 \\
     p_2^{\star} =-t  \\
    \end{cases}
\end{align}
与$p_2\geq0$的假设矛盾,不可取。

若式(1.29)第一项等于0,则:
\begin{align}
    (\frac{2t}{p_2^{\star}-p_1+t})^a-\frac{(a+1)p_2^{\star}-p_1+t}{p_2^{\star}-p_1+t} = 0
\end{align}
将式(1.32)与式(1.28)联立,得:
\begin{align}
&p_1=\frac{p_2+t}{a+1}\\
&p_2=(a+1) p_1-t  \\
&\left(\frac{2 t}{p_2-p_1+t}\right)^{a}-\frac{(a+1) p_2-p_1+t}{p_2-p_1+t}=0 \notag\\
&\left(\frac{2 t}{a p_1}\right)^a-\frac{a p_2+a p_1}{a p_1}=0 \notag\\
&\left(\frac{2 t}{a p_1}\right)^a-\frac{a\left[(a+1) p_1-t\right]+a p_1}{a p_1}=0\notag\\
&\left(\frac{2 t}{a p_1}\right)^a-\frac{(a+2) p_1-t}{p_1}=0\notag\\
&\left(\frac{2 t}{a p_1}\right)^a+\frac{t}{p_1}=a+2 \notag\\
&\left[\frac{(a+1) 2 t}{\left.a (p_2+t\right)}\right]^a+\frac{(a+1) t}{p_2+t}=a+2 
\end{align}
所以,均衡价格$(p_1^{\star},p_2^{\star})$等于:
\begin{align}
    \begin{cases}
    & p_1^* \in\left\{p_1:\left(\frac{2 t}{a p_1}\right)^a+\frac{t}{p_1}=a+2 ; \quad p_1 \geqslant 0, a>0\right\} \\
     &p_2^* \in\left\{p_2:\left(\frac{2}{a}\right)^a\left(\frac{(a+1) t}{p_2+t}\right)^a+\frac{(a+1) t}{p_2+t}=a+2 ; p_2\geqslant 0, a>0\right\}\\
    \end{cases}
\end{align}

\qsubpart
同样的,在$v_1=v_2$条件下,假设位于$x$的消费者在厂商1 和厂商2的消费效用相同,则有:
\begin{align}
     u_{1,x}&=u_{2,x} \notag \\ 
     v_1-tx-p_1(x) &= v_2-t(1-x)-p_2(x) \notag \\
     \therefore \  x &= \frac{p_2(x)-p_1(x)+t}{2t}
\end{align}
$p_i(x)$的假设与第二问相同。


则厂商1和厂商2的利润$\pi_1$和$\pi_2$满足:
\begin{align}
    \pi_1 &=\int_0^x p_1(x)f(x)dx \\
    \pi_2 &= \int_x^1 p_2(x)f(x)dx 
\end{align}
不妨设$F_1(x)$和$F_2(x)$满足:
\begin{equation*}
    H_i^{'} (x) = p_i(x)f(x) = h_i(x)
\end{equation*}
则式(1.7)和式(1.8)可以写为:
\begin{align}
    \pi_1(x) &=H_1(x)-H_1(0) \\
    \pi_2(x) &=H_2(1)-H_2(x)
\end{align}
利润最大化下,应满足:
\begin{align}
    \frac{\partial \pi_i(x) }{\partial p_i} &=\frac{\partial \pi_i(x)}{\partial x}\bigg/\frac{\partial p_i(x)}{\partial x}=0 
\end{align}
分母$\frac{\partial p_i(x)}{\partial x}$不为0,因此有:
\begin{align}
    \frac{\partial \pi_i(x)}{\partial x} &=0 \notag\\
    \therefore &\begin{cases}
      \frac{\partial \pi_1(x)}{\partial x} = h_1(x) =p_1(x)x^{a-1}=0  \\
      \frac{\partial \pi_2(x)}{\partial x} = -h_2(x) =p_2(x)x^{a-1}=0
\end{cases}  \notag \\
    \because & x^{a-1} \neq 0  \ \therefore p_i(x) =0\notag \\
    \therefore x &= \frac{1}{2} 
\end{align}
又因为$p_2$在$[0,\frac{1}{2})$上单调递增且,$p_2(x)=0$,所以在$[0,\frac{1}{2})$上,$p_{2,x}\equiv0$;同理,在$(\frac{1}{2},1]$上,$p_{1,x}\equiv0$。

因此,在$[0,\frac{1}{2})$上有:
\begin{align}
    u_1 \geq u_2 & \ and \ p_{2,x}=0 \notag \\
    v_1-tx-p_1(x) &\geq v_2-t(1-x)-p_2(x) \notag \\
    v_1-tx-p_1(x) &\geq v_2-t(1-x) \notag \\
   0\leq p_1(x) & \leq t-2tx 
\end{align}
为使利润尽可能地高,厂商应尽量提高价格因此有:
\begin{equation}
    p_1(x) = t-2tx 
\end{equation}

同理,在$(\frac{1}{2},1]$上,厂商2的价格满足:
\begin{equation}
    p_2(x) = -t+2tx 
\end{equation}

因此,在均衡时,价格$(p_1^{\star},p_2^{\star})$为:
\begin{equation}
p_1^{\star}(x)=
    \begin{cases}
      &  t-2tx,  \ x \in [0,\frac{1}{2}] \\
     &0 ,\  x \in ( \frac{1}{2},1]
    \end{cases}
\end{equation}
\begin{equation}
p_2^{\star}(x)= 
    \begin{cases}
     & 0,  \ x \in [0,\frac{1}{2}] \\
     & -t+2tx ,\  x \in ( \frac{1}{2},1]
    \end{cases}
\end{equation}

\qsubpart
对于消费者x所能接受的最大价格取决于效用,而其效用与其分布无关,因此与第三问相同,最大价格$p^m_{1,x}$,$p^m_{2,x}$等价于式(1.18)、式(1.19)。

同样的,假设第一问的消费者剩余为$S^{\alpha}$,均衡点(消费者效用无差异)为$x_0^{\alpha}$,均衡价格为$(p_1^{\star(\alpha)},p_2^{\star(\alpha)})$;第二问的消费者剩余为$S^{\beta}$,均衡点)为$x_0^{\beta}$,均衡价格为$(p_1^{\star(\beta)},p_2^{\star(\beta)})$:

\begin{align}
    S^{\alpha} &= \int_0^{x_0^{\alpha}}(p_1^m-p_1^{\star(\alpha)})f(x)dx+\int_{x_0^{\alpha}}^{1}(p_2^m-p_2^{\star(\alpha)})f(x)dx \\
    S^{\beta} &= \int_0^{x_0^{\beta}}(p_1^m-p_1^{\star(\beta)})f(x)dx+\int_{x_0^{\beta}}^{1}(p_2^m-p_2^{\star(\beta)})f(x)dx 
\end{align}

将式(1.25)、式(1.33-34)、式(1.36)、式(1.43)及(1.47-48)代入式(1.49)与式(1.50)整理得到:
\begin{align}
S^\alpha &=\int_0^{x_0^\alpha}\left(p_1^m-p_1^{\star(\alpha)}\right) f(x) d x+\int_{x_0^\alpha}^1\left(p_2^m-p_2^{\star(\alpha)}\right) f(x) d x \notag\\
&=\int_0^{x_0^\alpha}\left(v_1-t x-p_1^{\star(\alpha)}\right) d F(x)+\int_{x_0^\alpha}^1\left(v_2-t+t x-p_2^{\star(\alpha)}\right) d F(x) \notag\\
&=\left.\left(v_1-t x-p_1^{\star(\alpha)}\right) F(x)\right|_0 ^{x_0^\alpha}-\int_0^{x_0^\alpha} F(x) d\left(v_1-t x-p_1^{\star(\alpha)}\right)+\left.\left(v_2-t+t x-p_2^{\star(\alpha)}\right) F(x)\right|_{x_0^\alpha} ^1 \notag\\
& \quad-\int_{ x_0^\alpha}^1 F(x) d\left(v_2-t+t x-p_2^{\star(\alpha)}\right)\notag \\
&=\left(v_1-t x_0^\alpha-p_1^{\star(\alpha)}\right)\left(x_0^\alpha\right)^a+v_2-t+t-p_2^{\star(\alpha)}-\left(v_2-t+t x_0^\alpha-p_2^{\star(\alpha)}\right)\left(x_0^{\alpha}\right)^a\notag \\
&=v_2-p_2^{\star (\alpha)}+\frac{2 t\left(\frac{a p_1^{\star(\alpha)}}{2 t}\right)^{a+1}-t}{a+1} \notag \\
&=v_2-(a+1) p_1^{\star(\alpha)}+t+\frac{\left(\frac{a p_1^{\star(\alpha)}}{2 t}\right)^a a p_1^{\star(\alpha)}}{a+1}-\frac{t}{a+1} \notag \\
&=v_2+\frac{a t}{a+1}+\left[\frac{a p_1^{\star(\alpha)}}{p_1^{\star(\alpha)}+p_2^{\star(\alpha)}} p_1^{\star(\alpha)}-(a+1)^2 p_1^{\star(\alpha)}\right] \bigg/(a+1)
\end{align}

\begin{align}
 S^b&=\int_0^{x_0^\beta}\left(p_1^m-p_1^{\star(\beta)}\right) f(x) d x+\int_{x_0^\beta}^1\left(p_2^m-p_2^{\star(\beta)}\right) f(x) d x \notag\\
&=\int_0^{1 / 2}\left(v_1-t x-t+2tx\right) d F(x)+\int_{1 / 2}^1\left(v_2+t x-t+t-2 t x\right) d F(x) \notag\\
&=\left.\left(v_1-t+t x\right) F(x)\right|_0 ^{1 / 2}+\left.\left(v_2-t x\right) F(x)\right|_{1 / 2} ^1-\int_0^{1 / 2} F(x) d\left(v_1+t x-t\right)-\int_{1 / 2}^1 F(x) d\left(v_2-t x\right) \notag\\
&=\left(\frac{1}{2}\right)^a\left(v_1-v_2-\frac{t}{2}+\frac{t}{2}\right)+v_2-t-t \int_0^{1 / 2} x^a d x+t \int_{1 / 2}^1 x^a d x \notag\\
&=v_2-t+t \frac{1-\left(\frac{1}{2}\right)^{a+1} \times 2}{a+1}\notag\\
&=v_2-\frac{a t}{a+1}-\frac{t2^{-a}}{a+1} 
\end{align}

当消费者总剩余减少时,有$S^b<S^a$:
\begin{align}
&S^a-S^b=\frac{2 at+2^{- a}t}{a+1}+\left[\frac{a p_1^{\star(\alpha)}}{(a+2) p_1^{\star(\alpha)}-t} p_1^{\star(\alpha)}-(a+1)^2 p_1^{\star(\alpha)}\right] /(a+1)>0 \notag\\
&=2 a t+2^{-a} t+\frac{a p_1^{\star(\alpha)}^2}{(a+2) p_1^{\star(\alpha)}-t}-\left(a+1)^2 p_1^{\star(\alpha)}>0\right\notag\\
&=\left(2 a+2^{-a}\right) \frac{t}{p_1^{\star(\alpha)}}+\frac{a}{(a+2)-\frac{t}{p_1^{\star(\alpha)}}}-(a+1)^2>0 \notag\\
& Let \  \frac{t}{p_1^{\star(\alpha)}}=m \notag\\
&=\left(2 a+2^{-a}\right) m+\frac{a}{(a+2)-m}-(a+1)^2>0 \notag\\
&=-\left(2 a+2^{-a}\right) m^2+\left[\left(2 a+2^{-a}\right)(a+2)+(a+1)^2\right] m+\left[a-(a+1)^2(a+2)\right]>0   \\
&\therefore m\in \left\bigg(\frac{1}{2\left(-2 a-2^{-a}\right)}\left(-(a+1)^2-(a+2)\left(2 a+2^{-a}\right)-\right.\notag\\
&\left.\sqrt{\left((a+1)^2+(a+2)\left(2 a+2^{-a}\right)\right)^2-4\left(-2 a-2^{-a}\right)\left(a-(a+1)^2(a+2)\right)}\right), \notag \\
&\frac{1}{2\left(-2 a-2^{-a}\right)}\left(-(a+1)^2-(a+2)\left(2 a+2^{-a}\right)+\right. \notag\\
&\left.\sqrt{\left((a+1)^2+(a+2)\left(2 a+2^{-a}\right)\right)^2-4\left(-2 a-2^{-a}\right)\left(a-(a+1)^2(a+2)\right)}\right)\right\bigg)
\end{align}
即当$t$满足:
\begin{align}
   t &>\frac{p_1^{\star(\alpha)} }{2\left(-2 a-2^{-a}\right)}\left(-(a+1)^2-(a+2)\left(2 a+2^{-a}\right)-\right.\notag\\
&\left.\sqrt{\left((a+1)^2+(a+2)\left(2 a+2^{-a}\right)\right)^2-4\left(-2 a-2^{-a}\right)\left(a-(a+1)^2(a+2)\right)}\right) 
\end{align}
且
\begin{align}
   t &<\frac{p_1^{\star(\alpha)} }{2\left(-2 a-2^{-a}\right)}\left(-(a+1)^2-(a+2)\left(2 a+2^{-a}\right)+\right.\notag\\
&\left.\sqrt{\left((a+1)^2+(a+2)\left(2 a+2^{-a}\right)\right)^2-4\left(-2 a-2^{-a}\right)\left(a-(a+1)^2(a+2)\right)}\right) 
\end{align}
此时,消费者总剩余减少。


















